\begin{table}[!ht]
\caption{{\bf Selected examples of Brainhack projects}\label{tab3}}
\hrule %\hline
    \begin{itemize}
    \item 
        A child psychiatrist and a 3D video artist initiated a collaboration at the 2012 Brainhack to develop a movie to be shown to participants during resting-state fMRI scans to reduce head motion in hyperkinetic populations \cite{inscapes, vanderwal2015}.
    \item 
        The ABIDE Preprocessing Initiative \cite{abide_preproc} is an ongoing project started at the 2012 Brainhack to share preprocessed versions of the Autism Brain Imaging Data Exchange (ABIDE) dataset \cite{abide, dimartino2014}. This project is sharing functional data that have been processed using the Connectome Computation System (CCS) \cite{ccs, Xu2015}, the Configurable Pipeline for the Analysis of Connectomes (C-PAC) \cite{cpac, craddock2013}, the Data Preprocessing Assistant for Resting State fMRI (DPARSF) \cite{dparsf, yan2010}, and the Neuro Imaging Analysis Kit (NIAK) \cite{niak, bellec2011}, as well as cortical thickness measures extracted from structural data using FreeSurfer \cite{freesurfer, fischl2000}, CIVET \cite{civet, zijdenbos2002}, and Advanced Normalization Tools (ANTS) \cite{ants, tustison2014}.
    \item
        A collaboration started at the 2012 Brainhack performed an analysis to identify differences in cortical thickness and structural covariance between individuals with autism spectrum disorder and neurotypical controls \cite{Valk2015}.
    \item
        A project team at Brainhack 2013 amassed a dataset of 14,781 structural MRI scans to estimate the distribution of brain sizes across individuals for optimizing scan acquisition parameters \cite{Mennes2014}.
    \item 
        The development team of LORIS, an open source database system for neuroimaging and phenotypic data, have repeatedly used Brainhack as an opportunity to meet and collaborate on new features \cite{Das2012}.
    \item
        An early version of the Daydreaming app \cite{dayapp}, an Android application for real-time assessment of users' mind-wandering, was developed at Brainhack 2013.
    \item 
        The Clubs of Science \cite{clubs} project, founded at Brainhack MTL 2015, has built a web-based visualization of the social web underlying neuroimaging research.
    \item
        The linkRbrain \cite{linkrbrain} tool for integrating and querying neuroimaging data with activation peaks from the literature and gene expression data  was partially developed and first tested at Brainhack 2013 in Paris \cite{Mesmoudi2015}. 
    \end{itemize}
\hrule %\hline
\smallskip Further projects can be found at www.brainhack.org \cite{brainhackorg}.
\end{table}
