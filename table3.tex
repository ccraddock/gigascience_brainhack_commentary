\begin{table}[!ht]
\caption{{\bf Selected examples of Brainhack projects.} Further projects can be found at \href{http://www.brainhack.org}{brainhack.org}}\label{tab3}
\hline
    \begin{itemize}
    \item 
        A child psychiatrist and a 3D video artist initiated a collaboration at the 2012 Brainhack to develop a movie to be shown to participants during resting-state fMRI scans to reduce head motion in hyperkinetic populations \href{http://vimeo.com/67962604}{(video link)}\cite{vanderwal2015}.
    \item 
        The \href{http://preprocessed-connectomes-project.github.io/abide}{ABIDE Preprocessing Initiative} is an ongoing project started at the 2012 Brainhack to share preprocessed versions of the \href{http://fcon_1000.projects.nitrc.org/indi/abide}{Autism Brain Imaging Data Exchange (ABIDE)} dataset. This project is sharing functional data that have been processed using the \href{http://lfcd.psych.ac.cn/ccs.html}{Connectome Computation System (CCS)}, the \href{http://fcp-indi.github.io}{Configurable Pipeline for the Analysis of Connectomes (C-PAC)}, the \href{http://rfmri.org/DPARSF}{Data Preprocessing Assistant for Resting State fMRI (DPARSF)}, and the \href{https://www.nitrc.org/projects/niak/}{Neuro Imaging Analysis Kit (NIAK)}, as well as, cortical thickness measures extracted from structural data using \href{http://freesurfer.net/}{FreeSurfer}, \href{http://www.bic.mni.mcgill.ca/ServicesSoftware/CIVET}{CIVET}, and \href{http://picsl.upenn.edu/software/ants/}{Advanced Normalization Tools (ANTS)}.
    \item
        A collaboration started at the 2012 Brainhack performed an analysis to identify differences in cortical thickness and structural covariance between individuals who suffer from autism spectrum disorder and healthy controls\cite{Valk2015}.
    \item
        A project team at Brainhack 2013 amassed a dataset of 14,781 structural MRI scans to estimate the distribution of brain sizes across individuals for optimizing scan acquisition parameters\cite{Mennes2014}.
    \item 
        The development team of LORIS, an open-source database system for neuroimaging and phenotypic data, have repeatedly used brainhack as an opportunity to meet and collaborate on new features\cite{Das2012}.
    \item
        An early version of the \href{http://daydreaming-the-app.net}{Daydreaming app}, an Android application for real-time assessment of users mind-wandering was developed at Brainhack 2013.    
    \end{itemize}
\hline
\end{table}
