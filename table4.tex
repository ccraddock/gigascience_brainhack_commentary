\begin{table}[!ht]
\caption{{\bf Programming components of Brainhack events.}\label{tab4}}
\hline
    \begin{itemize}
    \item
        \textbf{Meet and greet:} Brainhack events begin with a welcome to the hosting facility by the local organizing committee, along with a briefing about the event schedule, procedures, or other information that might be important for the attendees.
    \item
        \textbf{Ice breaker:} Interaction between attendees is the key to a successful Brainhack event. The ice breaker is an activity to introduce attendees and their interests on one another. One strategy that has been successful is for each attendee to give their name, their institution, and three words that describe their interests. Such ice breakers could take various forms, including a speed-dating paradigm in which attendees pair-up for a brief conversation, after which attendees swap partners, and this continues until every pair has met.
    \item
        \textbf{Ignite talks:} Brainhack's equivalent of keynote sessions, Ignite talks are inspirational talks on the big picture of open brain science that are intended to invigorate the audience for the day ahead. These are ideally brief (10 minute presentation followed by 10 minutes of questions), of general relevance, and are provided by a luminary in the field.
    \item
        \textbf{Hacking:} The core of Brainhack is ``open hacking'' sessions during which attendees collaborate together on projects of their choosing. Attendees who have specific project ideas or data that they would like to explore are encouraged to advertise their project at \href{http://www.brainhack.org}{www.brainhack.org} prior to the event. On the opening of the Brainhack event, typically after the ice breaker, attendees pitch their ideas and afterwards mingle with others to organize a project team. Teams work together throughout the remainder of Brainhack and are given the opportunity to present their progress during the wrap-up session at the end of the event.
    \item
        \textbf{Brainhacking 101:} The educational track enables less experienced attendees to learn basic software and data analysis skills. Occuring in parallel so as not to interfere with the ongoing hacking sessions, this track begins with ``Installfest'' sessions during which attendees receive help installing needed software. Afterwards are several hands-on tutorials that cover topics like: using Github, Python programming, using Python to load and visualize neuroimaging data, and performing meta-analyses of scientific literature. The resources for educational sessions are made freely available online (e.g., see \href{https://github.com/ohbm/brain-hacking-101}{github.com/ohbm/brain-hacking-101}).
    \item
        \textbf{Unconference:} Sessions for attendees to present on their research or other topics of current interest. Immediately prior to these sessions, the agenda is determined on-site. Attendees who are willing to present add their name to a sign-up sheet and in the event that there are more interested presenters than time, the group is polled to determine which presentations are given time or to extend the amount of time allotted. Instead of unconference sessions, some sites have incorporated ``Data Blitzes'', consisting of preorganized session where attendees have the opportunity to present their research. Brainhack Miami has had success with this model and has secured funding to award monetary prizes to the best presentations.
    \item
        \textbf{Wrap-up and feedback:} Brainhack events typically finish with a wrap-up session during which project teams describe the progress that they made or give a demo of their results in a brief (~1-2 minute) presentation. Afterwards, the local organizers lead a discussion about what worked well with the event, and how it could be improved in the future.
    \end{itemize}
\hline
\end{table}